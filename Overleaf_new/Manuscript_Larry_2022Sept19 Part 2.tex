%2multibyte Version: 5.50.0.2960 CodePage: 936
%\usepackage{subfig}
%\usepackage{bbm}
%\hypersetup{hidelinks} % hide the ugly red box in footnote links
%\setcitestyle{authoryear,round}
%\doublespacing
%\graphicspath{{figure/}}
% defines ExpandableInput command which solves the
% problem of having noalign problem in the first line
% of input tables
% defines
% the superscripts in tables (significance stars)


\documentclass[12pt]{article}
%%%%%%%%%%%%%%%%%%%%%%%%%%%%%%%%%%%%%%%%%%%%%%%%%%%%%%%%%%%%%%%%%%%%%%%%%%%%%%%%%%%%%%%%%%%%%%%%%%%%%%%%%%%%%%%%%%%%%%%%%%%%%%%%%%%%%%%%%%%%%%%%%%%%%%%%%%%%%%%%%%%%%%%%%%%%%%%%%%%%%%%%%%%%%%%%%%%%%%%%%%%%%%%%%%%%%%%%%%%%%%%%%%%%%%%%%%%%%%%%%%%%%%%%%%%%
\usepackage{eurosym}
\usepackage{rotating}
\usepackage{amsmath}
\usepackage{graphicx}
\usepackage{verbatim}
\usepackage{caption}
\usepackage{subcaption}
\usepackage{setspace}
\usepackage{color}
\usepackage{amsfonts}
\usepackage{amssymb}
\usepackage{float}
\usepackage[authoryear,round]{natbib}
\usepackage{hyperref}
\usepackage{makecell}
\usepackage[toc,page,header]{appendix}
\usepackage{minitoc}
\usepackage[margin=1truein]{geometry}

\setcounter{MaxMatrixCols}{10}
%TCIDATA{OutputFilter=LATEX.DLL}
%TCIDATA{Version=5.50.0.2960}
%TCIDATA{Codepage=936}
%TCIDATA{<META NAME="SaveForMode" CONTENT="1">}
%TCIDATA{BibliographyScheme=BibTeX}
%TCIDATA{LastRevised=Tuesday, September 20, 2022 09:00:43}
%TCIDATA{<META NAME="GraphicsSave" CONTENT="32">}

\graphicspath{{./Images}}
\hypersetup{
colorlinks=true,
linkcolor=blue,
anchorcolor=blue,
citecolor=blue}
\providecommand{\U}[1]{\protect\rule{.1in}{.1in}}
\pagestyle{plain}
\newtheorem{theorem}{Theorem}
\newtheorem{acknowledgement}[theorem]{Acknowledgment}
\newtheorem{algorithm}{Algorithm}
\newtheorem{axiom}{Axiom}
\newtheorem{case}{Case}
\newtheorem{claim}{Claim}
\newtheorem{conclusion}{Conclusion}
\newtheorem{condition}{Condition}
\newtheorem{conjecture}{Conjecture}
\newtheorem{corollary}{Corollary}
\newtheorem{criterion}{Criterion}
\newtheorem{definition}{Definition}
\newtheorem{example}{Example}
\newtheorem{exercise}{Exercise}
\newtheorem{lemma}{Lemma}
\newtheorem{notation}{Notation}
\newtheorem{problem}{Problem}
\newtheorem{proposition}{Proposition}
\newtheorem{remark}{Remark}
\newtheorem{solution}{Solution}
\newtheorem{summary}{Summary}
\onehalfspacing
\newenvironment{proof}[1][Proof]{\textbf{#1.} }{\ \rule{0.5em}{0.5em}}
\RequirePackage{threeparttable}
\RequirePackage{booktabs}
\RequirePackage{tabularx}
\makeatletter\let\ExpandableInput\@@input\makeatother
\def\sym#1{\ifmmode^{#1}\else\(^{#1}\)\fi}
%\input{tcilatex}
\begin{document}

\title{Does Pollution Affect Exports? -- Evidence from China\thanks{%
Thanks note here...}}
\author{Jie Bai\thanks{%
Department of Economics, Lingnan University, Hong Kong, E-mail: jiebai@ln.hk}
\quad Larry D. Qiu\thanks{%
Department of Economics, Lingnan University, Hong Kong, E-mail:
larryqiu@ln.edu.hk} \quad Junji Xiao\thanks{%
Department of Economics, Lingnan University, Hong Kong, E-mail:
junjixiao@ln.edu.hk. \copyright\ 2022 Jie Bai, Larry D. Qiu and Junji Xiao.
All Rights Reserved. }}
\date{September 10, 2022}
\maketitle

\begin{abstract}
Trade is an important driving force for rapid economic growth but almost
inevitably accompanied by deterioration in environment, especially in
developing countries. In contrast to existing studies that investigate the
impact of trade on environment, this paper analyzes the reverse effects,
that is, whether pollution affects exports and if yes, how, by examining the
Chinese export and pollution data over the period of 2000--2007. Using $%
\mathrm{PM_{2.5}}$ concentrations as a proxy for air pollution, we estimate
the causal effect of air pollution on the Chinese firms' exports, employing
thermal inversion as an instrument variable. We find that an 1\% increase in 
$\mathrm{PM_{2.5}}$ leads to an 1.09\% reduction in firm-level exports. This
reduction in exports is mainly attributed to the intensive margins as
opposed to the extensive margins. The mechanism analysis suggests
coexistence of two channels: air pollution adversely affects productivity
and also invites stringent environmental regulations, both reducing exports.
\end{abstract}

\thispagestyle{empty}

\begin{quote}
\emph{Keywords:} Export; Air Pollution; Productivity; Environmental
Regulation

\emph{JEL Classification:} F14, F18, Q51
\end{quote}

\newpage \setcounter{page}{1}

\section{Introduction}

\label{sec:1} The rapid trade economic growth is usually accompanied by
environmental deterioration expecially in developing countries such as
China. From 1998 to 2007, air quality in China deteriorated substaintially.
For example, the average $\mathrm{PM_{2.5}}$ concentration increased from
32.32 $\mu g/m^{3}$ to 49.18 $\mu g/m^{3}$. The mean concentration exceeds
the WHO air quality guidelines every year, causing concerns about the
balance of environment and economic development.\footnote{%
For fine particulate matter ($\mathrm{PM_{2.5}}$), WHO published guideline
values for 5 $\mu g/m^{3}$ annual mean and 15 $\mu g/m^{3}$ 24-hour mean.
See %
\url{https://www.who.int/news-room/fact-sheets/detail/ambient-(outdoor)-air-quality-and-health}
for more information.} [\textbf{Question}: according to the footnote, there
are two types of mean. Which one did you report above?] Empirical evidences
have shown that severe pollution reduces labor productivity through
influencing people's health both physically and mentally as health is a very
important part of human capital %
\citep{graff2012impact,chang2016particulate,zhang2018impact,fu2021air,somanathan2021impact,Adhvaryu2022}%
. In this paper, we ask a different but related question: does pollution
influence international trade and how? We examine this question based on the
evidence of China, which experienced rapid increases in both trade and
pollution.

There exists a large literature of the impacts of trade on environment.%
\footnote{%
See \cite{cherniwchan2017trade} for the most recent survey of the literature.%
} However, studies on the impacts of environment on trade are rare. As
common, a biggest challenge in the empirical studies of trade and
environment is the endogeneity problem. This problem is more serious in the
estimation of the effects of pollution on trade than that of trade on
pollution, because in the latter case, researchers often find good
instrumental variables for trade, such as trade agreements and trade shocks.
In this study, we overcome the endogeneity problem using the instrumental
variable method to examine the causal impact of air pollution on firms'
exports. In particular, we follow the literature 
\citep{fu2021air,
khanna2021productivity, chen2022effect} to use both thermal inversions and
wind directions as instrumental variables for an important type of air
pollution, i.e., $\mathrm{PM_{2.5}}$ concentrations, at county level, to
study the effects of pollution on firms' exports in China from 2000 to 2007.

Our empirical findings suggest that exports decrease as $\mathrm{PM_{2.5}}$
levels increase. Specifically, following a 1\% increase in $\mathrm{PM_{2.5}}
$ concentration, the average exporter decreases its export value by 1.09\%.
Air pollution also influences firms' exporting decisions: an 1\% increase in 
$\mathrm{PM_{2.5}}$ concentration reduces a firm's probability of entering
foreign markets by 0.06 percentage point and raises an exporter's
probability of exiting foreign markets by 0.15 percentage point. We also
find that air pollution has detrimental effects on labor productivity and
total factor productivity (TFP), echoing the findings in the existing
literature \citep{greenstone2012effects,fu2021air,khanna2021productivity}.
This suggests that reducing productivity is a channel through which air
pollution reduces exports. In addition, we find that the negative effect of
air pollution on exports is significant only in the regions with
government-specified pollution abatement targets. This finding suggests
another channel: governments in pollution-intensive regions impose more
stringent control on pollution, raising the costs of production and so
undermining the firms' competitiveness in exports. This last finding
provides additional evidence supporting the previous studies on how
environmental regulations affect individual firms' exports %
\citep{cherniwchan2022international}. [\textbf{NOTE}: please revise this
reference: change the capital letters of the title to small letters. we
should also include the working paper no. Also check other reference
entries: not consistent in using capital letters and small letters, for
example: Journal of development economics. china, Wto, etc.]

This paper makes a significant contribution to the literature of
international trade and environment. Several review articles of this
literature are available including \cite{cherniwchan2017trade} and \cite%
{cherniwchan2022international} as the most recent ones. Almost all existing
studies focus on how international trade affect pollution. A general
conclusion is that trade liberalizatoin (and therefore trade) alters
emissions (or more generally, pollution) via scale, composition, and
technique effects (Grossman and Kruger, 1991; Copeland and Taylor, 1994).
The composition effect receives particular attention as it is related to the
pollution haven hypothesis, that is, pollution-intensive industries
\textquotedblleft move\textquotedblright\ to countries countries with lax
environment policies \citep{copeland1994north,taylor2005unbundling}. Our
paper clearly differs from this strand of literature as we examine the
opposite direction of the effects: how pollution affects trade.

Although they are scant, a few studies exist in examining the effects of
environmental factors on trade. \cite{jones2010climate} [\textbf{NOTE}: this
paper is not listed in the References. Please check the match: whether all
cited papers are included in the References and whether all papers in the
References are cited in our paper.] find that higher temperatures in poor
countries lead to large, negative impacts on the growth of their exports,
mainly in agricultural expoerts and light manufacturing exports. \cite%
{dellink2017international} verbally describe the potential channels through
which climate change damages affect trade. The channels include the direct
impact on transport and distribution chains, and the indirect influences on
factors of production. Unlike these papers, we conduct a rigorous empirical
analysis of the effects of air pollution on exports, based on firm-level
data. Thus, our study is a valuable addition to this strand of literature.

As mentioned earlier, the mechanisms that work in our finding of the
pollution effects on exports are productivity and environmental regulations.
This part of study in our paper is closely related to two literatures. The
first is the effects of pollution on productivity. [\textbf{To be added}]
Our finding is consistent with the general conclusion of this literature.

The second literature is the effects of enviromental regulations on exports.

[\textbf{NOTE to myself}: I need to understand how we estimate productivity
first. If productivity has include abatement cost, then both pollution and
regulation afffect productivity jointly, rather than pollution affects
productivity as one channel and regulation affects cost as another.]

Our study is also related to a small literature on the impacts of
environmental regulations and policies on various economic outcomes,
especially on productivity and trade. The effects of environmental
regulations on productivity are mixed. \cite{berman2001environmental} find
that TFP rises in air quality regulation regions as abatement investment
appear to be productivity enhancing in the US. \cite{greenstone2012effects}
find stricter air quality regulations are associated with a roughly 2.6
percent decline in TFP in the US. \cite{he2020watering} find that a water
quality regulation campaign in China significantly reduces upstream
polluting firms TFP by more than 24\% compared to downstream. They link this
water quality regulations to political evaluations. Therefore, local
government officials were more incentive to tighten the water quality
standards of upstream of monitor stations.

A few recent papers explore the environmental regulations and their
influences on trade.\footnote{%
Some papers investigate the effects of environmental regulations on foreign
direct investment \citep{dean2009foreign,cai2016does}.} \cite%
{hering2014environmental} study the Two Control Zones policy in China and
find that exports significantly reduce in targeted cities and sharper fall
for pollution-intensity industries. \cite{shi2018environmental} show that
pollution reduction targets in China's Eleventh Five-Year Plan result in
exports fall in targeted cities. Different from these studies, our paper
focuses on the effects of pollution \textit{per se} on exports.

The rest of the paper is orgnanized as follows. We specify our empirical
strategy in Section~\ref{sec:empirical_strategy}. We describe our data and
measurements in Section 3. We present the main empirical findings in Section
4 and explore the mechanisms in Section 5. We provide concluding remarks in
Section 6.

\section{Empirical Model and Identification}

\label{sec:empirical_strategy} This section presents our empirical
methodology of estimating the impact of air pollution on individual firms'
exports. Two stage least squares (TSLS) estimation is applied to solve the
potential endogeneity problem with pollution.

\subsection{Empirical Model}

To investigate the impact of air pollution on firms' exports, we assume
exports to be a function of pollution; moreover, we control for the major
determinants of exports following previous literature %
\citep[e.g.,][]{Kunst1989,Bernard2003}. Specifically, letting $Export_{ict}$
denote the value of exports by firm $i$ from county $c$ in year $t$, we
assume 
\begin{equation}
log(Export_{ict})=\beta _{0}+\beta _{1}log(P_{ct})+\mathbf{X}%
_{ict}^{^{\prime }}\Gamma +\mathbf{W}_{ct}^{^{\prime }}\beta _{w}+\delta
_{i}+\lambda _{t}+\epsilon _{ict}.  \label{equ1}
\end{equation}%
In the above equation, $log(Export_{ict})$ is the logarithmic value of
exports. $Log(P_{ict})$ is the logarithmic air pollution density. The vector 
$\mathbf{X}_{ict}^{^{\prime }}$ consists of variables measuring the firm
characteristics, including age, size, total capital, and productivity.
Productivity is measured by either labor productivity or revenue-based TFP.
Labor productivity is defined as the ratio of the firm's value added to the
number of employees. We estimate TFP using both Levinsohn-Petrin method %
\citep{levinsohn2003estimating} and Olley-Pakes method %
\citep{olley1996dynamics}, denoted as $TFP_{LP}$ and $TFP_{OP}$,
respectively. TFP is estimated using the output data at two-digit Chinese
Standard Industrial Classification (CSIC), considering heterogeneous
productivity across industries. All measures of productivity are in
logarithmic form and are deflated by respective price index developed by %
\citep{brandt2017wto}. As the previous literature %
\citep[e.g.,][]{jones2010climate} suggests that weather may affect exports,
we also control for weather, represented by vector $\mathbf{W}%
_{ct}^{^{\prime }}$, which includes temperature, precipitation, humidity,
wind speed, and sunshine duration. $\lambda _{t}$ is the time fixed effects,
capturing the time trends and all other possible time-specific shocks such
as annual socioeconomic cycles that may affect exports. $\delta _{i}$ is the
firm fixed effects, which is time-invariant, capturing firm-specific
unobservable characteristics contributing to exports. $\epsilon _{ict}$ is
the error term, which is assumed to be mean zero and independent across
firms, allowing within-firm auto correlation.

We list the definitions of the key variables in Table 1 for easy reference.

\begin{center}
$<$Table~\ref{tab:var_definition} Here$>$
\end{center}

\subsection{Identification}

\label{sec:identification} 
% Our outcome variable $Y_{ict}$ are about the intensive margin
% measurement of export: natural logarithm form of the export value of continuing
% exporting firm $i$ in county $c$ and year $t$\footnote{%
% In the way, we only keep export firms that is continuing in exporting from
% the year they started exporting to the last observation year in data.
% Another way to look at intensive margin is keep exporters and observation
% with positive export value, which we show in Appendix A.1.}, and the
% extensive margin of export: the entry \& exit decisions of export of firm $i$
% in county $c$ and year $t$.
One concern with this model identification is the endogeneity problem of
pollution: The determinants of exports affect firms' outputs and so the
emission of pollutants, resulting in a correlation between the independent
variable $Log(P_{ct})$ and the error term $\epsilon _{ict}$. Intuitively,
the increased exports and deteriorated environment could be the simultaneous
consequence of positive shocks to production. This endogeneity problem makes
ordinary least sqaures (OLS) estimators biased.\footnote{%
The OLS estimator could suffer from upward or downward bias. For example,
certain firms may upgrade their production technology over time and adopt
energy-saving and environmentally friendly technology. In this scenario, low
pollution and high exports are observed, attenuating the negative effects of
pollution on exports and so leading to upward bias. In the contrary, certain
firms may have insufficient funds to upgrade their production technology
over time. In this scenario, OLS estimator generates downward bias. Local
environmental regulations may also bias the OLS estimators. Regions with
more exports usually observe faster economic growth at the cost of
pollution. In response, the local governments may impose more stringent
environmental regulations, leading to low pollution but high production
costs and so low exports %
\citep{cherniwchan2022environmental,shi2018environmental}; as a consequence,
the OLS estimator will also suffer from downward bias.} [\textbf{Question}:
Should we also talk about the causality problem here?] To address this
problem, we use an IV for $log(P_{ct})$ and apply the TSLS estimation to
Model (\ref{equ1}). Following the literature of environmental economics %
\citep{arceo2016does,jans2018economic,sager2019estimating,fu2021air,khanna2021productivity}%
, we use county-level days of thermal inversions per annum as the
instrumental variable. In the first stage of the TSLS, we regress pollution
on TI, controlling for the variables as in Model~(\ref{equ1}) and get the
fitted value of $Log(P_{ct})$ to be used for the second stage of the TSLS.

As described by Jacobson (2002) [\textbf{NOTE}: include this paper in the
Reference: Jacobson, M. (2002). Atmospheric Pollution. History, Science, and
Regulation, Cambridge: Cambridge University Press.] and Arco, et al. (2016),
thermal inversion is a reversal of the normal behaviour of temperature in
the troposphere, which is the region of the atmosphere nearest Earth's
surface). Under normal conditions air temperature usually decreases with
height. When a layer of cool air at the surface of the troposphere is
overlain by a layer of warmer air, which acts as a cap on the upward
movement of the air from the layers below, a thermal inversion occurs (see
Fig.~\ref{fig:1}). Consequently, convection caused by the heating of the air
from below is limited to levels below the inversion and so air pollution is
trapped close to the ground due to thermal inversion. This positive
correlation between air pollution and thermal inversion is owing to
meteorological factors.

\begin{center}
$<$Figure~\ref{fig:1} Here$>$
\end{center}

Thermal inversion occurs for three reasons. First, a subsidence inversion
occurs when there is a high pressure system and air from high pressure
center in the atmosphere sinks down to fill the space left as air blows
outward, resulting in a layer of hot air on a layer of cold. Second, thermal
inversion can happen when air moves horizontally and a warm air layer is
squeezed upwards onto the cold denser air. Third, a radiation inversion
happens when air temperature near the ground cools down faster than that in
the air above, resulting in the temperature near the ground lower than that
in the upper layers. [\textbf{NOTE}: I don't understand the logic of this
paragraph. We say TI occurs for three reasons at the begining, but then the
first is about subsidence inversion and the last is about radiation
inversion. What are their relation to TI?]

Because thermal inversion is a pure meteorological phenomenon which affects
pollution but is unlikely to directly affect other economic and social
activities, it has become a popular instrumental variable for pollution in
empirical studies %
\citep{arceo2016does,jans2018economic,sager2019estimating,chen2022effect,khanna2021productivity}%
. It is also a valid intromental variable in our study because it satisfies
the following three conditions: Relevance, exogeneity and exclusion. First,
as we describe above, when a thermal inversion occurs, pollutants are
trapped in the air close to the ground, and thus, pollution level increases.
Hence, a number of thermal inversions observed in a year is positively
correlated with the average pollution level of the year. The relevance
condition is satisfied. We confirm this positive correlation later based on
our data of thermal inversions and $\mathrm{PM_{2.5}}$ at county-year level
in China.

\ Second, no evidence suggests that thermal inversions are correlated with
exports. Thermal inversions are driven by meteorological factors but not by
socioeconomic factors, and so they not correlated with economic activities
such as exports. Hence, our instrumental variable satisfies the endogeneity
condition. We confirm this later with the evidence based on our data of
thermal inversions at county-year level in China. Third, as discussed
previously [\textbf{Question}: where exactly?], the only channel through
which thermal inversions may affect exports is through pollution. 
% \subsection{IV Approach} \label{sec:2.2}

% The identification requires the $Export_{ict}$ is independent with $\epsilon
% _{ict}$ conditional on control variables. However, the effects of the air
% pollution on export may not be causal, because of export and pollution are
% simultaneous. That is, the increasing of trade corresponding to expanding of
% production companies with deteriorated environment. Absent any effect of
% pollution on exports, higher exports in a county would lead to both higher
% production and worse pollution. In this case, $\hat{\beta _{1}}$ in Eq.(\ref%
% {equ1}) may subject to bias upward. Fixed effects estimators exacerbate
% measurement error, biasing $\hat{\beta _{1}}$ further upward toward to zero.
% The other endogenous problem is omitted variable due to locally time-varying
% changes that lead to either underestimate or overestimate effects of
% pollution on exports in the OLS estimates. For example, counties within more
% high-tech firms may upgrade their production technology over time and
% implement more advanced productive energy-saving \& environmentally friendly
% technology, leading to upward bias. In the contrary, less productive firms
% may have insufficient funds to upgrade their production technology over
% time, leading to a downward bias as technology degrades. Trends in local
% environmental regulations also bias OLS estimates. Regions with more exports
% coincide with faster economic growth impose more stringent environmental
% regulations over time, leading to a downward bias. To identify the
% causality, we apply an instrumental variable approach. The instrumental
% variable is used to predict the $\rm PM_{2.5}$ concentrations across counties but has
% no direct effect on export. The endogenous variable $Log(PM_{2.5})_{ict}$ is
% instrumented by annual days with thermal inversion for each county to
% isolate the causal effect of export on pollution.

% We use the two-stage least squares (TSLS) to estimate: 
% \begin{equation} \label{equ2}
% Y_{ict} = \beta_{0} + \beta_{1} Log(PM_{2.5})_{ict} + \beta_{2}%
% \mathbf{W}_{ict} + \mathbf{X}^{^{\prime }}_{ict} \Gamma + \delta_{i} +
% \lambda_{t} + \epsilon_{ict}
% \end{equation}
% where the fitted value of $Log(PM_{2.5})$ for
% county $c$ in year $t$, is generated by the first-stage regression in the IV
% framework: 
% \begin{equation}  \label{equ3}
% Log(P_{ct}) = \alpha_{0} + \alpha_{1}TI_{ct} + \alpha_{2}\mathbf{W}%
% _{ct} + \mathbf{X}^{^{\prime }}_{ict}\Phi + \delta_{i} + \lambda_{t} + \mu{%
% ct}
% \end{equation}
% where, $TI_{ct}$ is the annual days of thermal inversion for
% county $c$ in year $t$. Standard error is clustered at firm level, but we
% also check the robustness of standard errors at county-year level pattern
% that allows spatial correlation within each county-year.

\begin{center}
$<$Figure~\ref{fig:2} Here$>$
\end{center}

\section{Data and Measurement}

\label{sec:3}

\subsection{Pollution}

\label{sec:3.1} Our study focuses on air pollution. There exist different
measures for air pollution. The US Environmental Protection Agency has
identified six pollutants as \textquotedblleft criteria\textquotedblright\
air pollutants, including carbon monoxide, lead, nitrogen oxides,
ground-level ozone, particle pollution (often referred to as particulate
matter), and sulfur oxides. These pollutants are found to harm human health
and the environment. Among these pollutants, the particulate pollutant that
is 2.5 microns or smaller in size, or $\mathrm{PM_{2.5}}$, is widely used in
the literature as an index for air pollution %
\citep{fu2021air,khanna2021productivity}. Following this literature, we also
use $\mathrm{PM_{2.5}}$ concentration to measure air pollution.

We use the satellite-based $\mathrm{PM_{2.5}}$ concentrations constructed
and estimated by \cite{van2021monthly} in our study.\footnote{\cite%
{van2021monthly} estimate fine particulate matter ($\mathrm{PM_{2.5}}$) by
combining Aerosol Optical Depth (AOD) retrievals from the NASA MODIS, MISR,
and SeaWIFS instruments with the GEOS-Chem chemical transport model, and
subsequently calibrating to global ground-based observations using a
Geographically Weighted Regression (GWR) at high grid resolution. The Global
Annual $\mathrm{PM_{2.5}}$ Grids data is available on the website of the
Atmospheric Composition Analysis Group at Washington University in St. Louis
( \url{https://sites.wustl.edu/acag/datasets/surface-pm2-5/}).} The
concentrations are available at a spatial resolution of 0.01- by-0.01-degree
latitude-longitude grid of the GIS raster (approximately 1.11 kilometers by
1.11 kilometers). To the best of our knowledge, this is the finest
resolution for the concentration data. This satellite-based data has a few
advantages over ground-based pollution data from official sources. First,
the satellite-based $\mathrm{PM_{2.5}}$ cover all counties, the smallest
administrative unit in China, and are available from 1998, whereas the
official $\mathrm{PM_{2.5}}$ data are only available for large cities since
2012.\footnote{%
On February 29, 2012, the State Council of China firstly agreed to issue the
newly revised Ambient Air Quality Standards. The new standard adds
monitoring indicators for fine particulate matter ($\mathrm{PM_{2.5}}$) and
ozone ($O_{3}$ ). Therefore, the official data of $\mathrm{PM_{2.5}}$
concentration was not available until then.} Second, official air quality
data may be potentially manipulated by local officials before publication %
\citep{ghanem2014effortless,andrews2008inconsistencies}, because local
environmental quality has become a vital criterion for government officials%
\^{a}\euro \texttrademark\ promotion evaluation since 2006 and particularly
after 2015 \citep{Fan2022}.

We aggregate grid-level $\mathrm{PM_{2.5}}$ concentrations to county level.
Specifically, we obtain a county's annual $\mathrm{PM_{2.5}}$ concentration
using the mean value of $\mathrm{PM_{2.5}}$ concentrations from all grid
cells within the county. [\textbf{NOTE}: However, you have not describe how
to obtain the \textbf{annual} concentration of each \textbf{grid}.] Figure~%
\ref{fig:3} presents the map of county-level average concentration of $%
\mathrm{PM_{2.5}}$ over the period 1998--2007 in China. There exist
substantial spatial variations of $\mathrm{PM_{2.5}}$ concentrations across
counties, ranging from 1.33 $\mu g/m^{3}$ to 97.57 $\mu g/m^{3}$. [\textbf{%
NOTE}: I don't find the following statements convincing and helpful. Should
delete it: \emph{The northeastern and eastern regions and southern Xinjiang
areas, an autonomous territory in northwest China, suffer from more severe
air pollution. As most of these areas also observe higher economic growth
rates than the other inland areas during the same period,\footnote{%
The economic growth in northeast of China and Xinjiang is lower than the
other areas. Their air pollution mainly stems from coal-fired power plants.
Moreover, coal mining, vehicle emissions, industrial factories, and
underground coal fires also contribute to its heavy air pollution. For
example, Xinjiang provides the rest of China with 40.6 percent of its coal
supply. Its capital city, Urumqi, is the number one coal-consuming city in
China.} this suggests that economy may have developed at the cost of
environment in China}.] The spatial heterogeneity in air pollution could be
attributed to differences in physiographic and meteorological conditions
across regions on the one hand, and also to the differences in regional
socioeconomic factors such as economic activities and environmental
regulations on the other hand. The large variations of $\mathrm{PM_{2.5}}$
concentrations give us the opportunity to explore the pollution effects on
economic activities such as exports.

\begin{center}
$<$Figure~\ref{fig:3} Here$>$
\end{center}

\subsection{Thermal Inversions and Weather}

\label{sec:data_TI} Thermal inversion is an phenomenon of air temperature at
the near ground second layer being higher than the first layer. To measure
the occurence of thermal inversions, we exploit the vertical temperature
data from NASA Modern-Era Retrospective Analysis for Research and
Applications (Version 2, or MERRA-2).\footnote{%
This dataset is titled M2I6NPANA and available at %
\url{https://disc.gsfc.nasa.gov/datasets/M2I6NPANA_5.12.4/summary}.} This
data set provides 6-hour air temperature at 42 atmospheric layers at a grid
level of 0.5 
%TCIMACRO{\U{b0} }%
%BeginExpansion
${{}^\circ}$
%EndExpansion
by 0.625 
%TCIMACRO{\U{b0} }%
%BeginExpansion
${{}^\circ}$
%EndExpansion
(about 45km by 55km in terms of distance). For each grid cell, we compare
the air temperature in the ground atmospheric layer (in the surface measure
conditions of 1000 Hectopascal pressure unit (hPa), which is around 110
meters above sea level), called layer 1, with that in the second atmospheric
layer close to ground (in the measure conditions of 975 hPa, approximately
320 meters above sea level), called layer 2.\footnote{%
Sometimes we observe missing temperatures at low air pressure levels because
the altitude of the land surface is high in that grid. In those cases, we
label the atmospherical layers based on their relative hight to the surface,
rather than using the uniform air pressure levels. As a result, layer 1
always corresponds to the lowest air pressure level above the surface.}
Under normal conditions, temperature decreases with altitude. If on a
particular day we observe that layer 1's temperature higher than layer 2's
temperature, then we say thermal inversion occures on that day. We obtain
the measure of thermal inversions of each grid cell using the number of days
with thermal inversions occurrence in that grid cell in a year. [\textbf{NOTE%
}: I don't understand the following description: Then, we \textbf{aggregate}
the grid level TIs to the county average using the cell areas spanning the
county as \textbf{weights}.] In the robustness check, we define another
measure of thermal inversions by comparing air temperature differences
between the ground and third-close-to-ground (the 950hPa layer, which is
around 540m above sea level) layers.

We obtain station-level daily weather indicators from National
Meteorological Information Center of China. In particular, we make use of
the dataset is provided by National Tibetan Plateau Data Center %
\url{(http://data.tpdc.ac.cn)}, which contains daily basic meteorological
variables including air temperature, precipitation, relative humidity, wind
direction and speed, sunshine duration and barometric pressure, collected
from 699 national meteorological stations in China since year 1951 %
\citep{Dailymeteorologicaldataset}. For each county, we use
Inverse-Distance-Weighting (IDW) method to convert the station-level
observations into county-level data, assigning lower weights to the stations
further away from the county center. Temperature is a key indicator of
weather. We use a vector of temperature distribution as the temperature
variable of each county. [\textbf{NOTE}: I don't understand the following
description: \emph{Taking into account the nonlinear effects of weather as
proposed by \citep{deschenes2017defensive}, we divide the sample
distribution of temperature into twenty equal-size quantiles and then count
the frequency of the daily temperature (the number of days) falling in these
quantiles for each county each year. In this way, we generate a
twenty-dimensional vector recording the frequency distribution of the
county's temperature}.] [\textbf{Question}: which papers use such type of
temperature variable?]

We also include the squared terms of precipitation, humidity, wind speed,
and sunshine duration, respectively, as weather controls. [\textbf{Question}%
: why using squared terms rather than distribution like temperature? Some
explanations.] All these individual variables together form the weather
vector in $\mathbf{W}_{ct}^{^{\prime }}$ Model (\ref{equ1}).

With the above constructions of county-level $\mathrm{PM_{2.5}}$
concentrations and thermal inversions, we can now explore their correlations
and time trends, which are presented in Fig.~\ref{fig:2}. The top panel
shows a strong positive correlation between thermal inversions and $\mathrm{%
PM_{2.5}}$ concentrations: counties that had high average thermal inversions
over 1998--2007 also had high average $\mathrm{PM_{2.5}}$ concentrations
over the same period. This is a supporting evidence of thermal inversions'
satisfaction of the relevance condition for a valid instrumental variable,
as argued earlier.

The bottom panel shows the time trends of thermal inversions and $\mathrm{%
PM_{2.5}}$ over the period of 1998--2007. First, there is no obvious time
trend of thermal inversions, which is largely consistent with the definition
of thermal inversions being an meteorological phenonenon not related to
economic activities and pollution. This is a supporting evidence of thermal
inversions' satisfaction of the exogeneity condition for a valid
instrumental variable, as argued earlier.

[\textbf{Question}: Any additional information this can provide compared to
the top panel? Moreover, the two curves of PM2.5 imply an increasing time
trend, which is true. But this may creat unncesarry question: TI is random
(no time trend) but PM2.5 has time trend, then are they positively
correlated? Think more carefully about this.] Second, $\mathrm{PM_{2.5}}$
concentrations the air pollution is systematically higher in the areas with
high TI frequencies than in the areas with low TI frequencies, suggesting a
positive correlation between TI occurrence and $\mathrm{PM_{2.5}}$
concentrations.]

\subsection{Firm-level Data}

\label{sec:data_firm} The firm characteristics are calculated based on data
collected from the Annual Survey of Industrial Firms (ASIF) by China's
National Bureau of Statistics. The ASIF dataset covers all firms with annual
sales above RMB 5 millions in China from 1998 to 2013. In total, there are
over 1.4 million observations for around 430 thousand firms over all
provinces in China. Following the literature \citep[e.g.,][]{fu2021trans},
we focus on firms in the manufacturing sector in empirical analysis.

The ASIF data provide detailed information of each firm's balance sheet,
such as gross output, value added, employment, capital, intermediate inputs,
and ownership. To correct the potential errors from data input, we drop the
abnormal observations that violate the following Generally Accepted
Accounting Principles: liquid assets are greater than total assets, total
fixed assets are greater than total assets, the net value of fixed assets is
greater than total assets, and current depreciation [\textbf{Question}:
depreciation of what?] is greater than accumulated depreciation.
Furthermore, we drop firms with obviously incorrect or missing establishment
time. We drop observations with missing or negative values for any one of
the following variables: output, value added, employment, and capital. We
drop firms with employees smaller than 8 following the previous research %
\citep{brandt2012creative}. As a result, given our sample period being
2000--2007, our firm dataset consists of 1,722,982 firm-year observations,
with 474,300 unique firms. Based on the firms' accounting information, we
derive measures of the firm characteristics variables for our analysis,
including firm age, which is defined as the number of years from the firm's
establishment up to the time of the survey, firm size, which is defined as
the number of employees, and capital, which is the real value of net fixed
assets. In our econometric analysis, all these variables are in logarithmic
forms, contained in vector $\mathbf{X}_{ict}^{^{\prime }}.$

Export data come from the Customs transactions database, which is available
from China's General Administration of Customs. The database provides
firm-level import and export information, which include both value and
quantity at HS six-digit code level. We just use export value in our
analysis. We aggregate each firm's export transactions to annual level to
obtain $Export_{ict}$. We merge Customs data with the ASIF data using firm's
names, telephone numbers, locations, etc. \citep{yu2015processing}. In our
merged data sample, there are around 17\% to 23\% of the firms exporting in
each year.

[\textbf{NOTE}: need a description about counties]

\subsection{Summary Statistics and Stylized Facts}

\label{sec:data_summary} We first present certain patterns and stylized
facts of the key variables in our sample. Figure~\ref{fig:4} exhibits the
patterns of air pollution and exports. Panel~\ref{fig:fig_a} presents the
time trends of the county-level average of logarithmic annual exports of
manufacturing firms in each county and the average of logarithmic annual $%
\mathrm{PM_{2.5}}$ of each county, which indicate that both average exports
and pollution have the upward time trends from 2000 to 2007. Panel~\ref%
{fig:fig_b} is a scatter plots of each county's annual average $\mathrm{%
PM_{2.5}}$ concentrations over the sample period (horizontal axis) and
annual average of export value over the sample period (vertical axis), which
indicate a negative correlation between $\mathrm{PM_{2.5}}$ and exports at
county level. This negative correlation serves as a preliminary evidence
that exports are lower in counties with higher $\mathrm{PM_{2.5}}$
concentrations. Of course, we cannot exclude the possibility that some
confounding factors affect these two variables in opposite ways and cause
their negative correlation across counties. In the subsequent sections,
therefore, we conduct rigorous analyses to control for those confounding
factors and investigate the causal relationship between air pollution and
exports.

\begin{center}
$<$Figure~\ref{fig:4} Here$>$
\end{center}

We next present the summary statistics of the key variables in Table~\ref%
{tab:var_definition}. In each county in each year, there are exporting firms
and non-exporting firms. The statistics are calculated based on exporting
firms only because our main analysis is about the pollution effects on the
intensive margins of exports. On average, there are 19 exporters per county
in each year. Our sample covers 1908 counties, which is accounted for around
67\% of the total number of counties in China (2864); and, it covers 96\% of
all the cities (321 out of 333 cities) and 91\% of all the provinces in
China.\footnote{%
Counties without export records are excluded from our sample.}

\begin{center}
$<$Table \ref{tab:stat} Here$>$

\bigskip 
\end{center}

\textbf{Part 2 starts Here}

\begin{center}
\bigskip 
\end{center}

\section{Empirical Results}

\label{sec:4}

In this section, we conduct empirical analyses of Model~(\ref{equ1}) and
discusses the estimation results. We first examine the impact of air
pollution on the intensive margins of exports using the sample of firms with
continuous exporting. Then, we analyze the effects of air pollution on the
firms' entry and exit decisions on exporting, i.e., the extensive margins.

\subsection{Effects of Air Pollution on Intensive Margins of Exports}

\label{sec:4.1}

Many firms in our ASIE dataset did not have any export in any year during
the sample period 2000--2007. Some firms had exports in certain years but
not in other years. Some firms had exports every year. To study intensive
margins, we following the common practice in the literature to use the
following subsample: all observations with a positive value of exports in
year $t$, where $t$ is from 2000 to 2007. This allows us to avoid the
problem of too many zeros in the sample.

We start our empirical analysis from the OLS regression based on Model (\ref%
{equ1}). We report the estimation results in columns (1)--(3) of Table~\ref%
{tab:main_results}. Column (1) includes only the pollution variable, $%
log(PM_{2.5})$, with firm FE and Year FE. Column (2) includes all firm-level
control variables, $\mathbf{X}_{ict}^{^{\prime }}$, with firm FE and Year
FE. Column (3) differs from column (2) by adding weather controls, $\mathbf{W%
}_{ct}^{^{\prime }}$. In both columns (2) and (3), productivity is
calculated using the LP method of TFP. [\textbf{NOTE}: sometimes you use $%
log $ and sometimes $Log$, in both main text and tables. Be consistent.] The
OLS results suggest that the values of firms' exports are negatively
correlated with $\mathrm{PM_{2.5}}$, with or without the controls of firm
characteristics and weather. The signs of firm characteristics are
consistent with those obtained and well known in the literature of
hetergenous firms' exports. In particular, high productivity firms export
more.

However, due to the endogeneity problem of $log(PM_{2.5})$, the estimates
from OLS regression are biased. TSLS is applied, employing thermal inversion
as an instrumental variable for $\mathrm{PM_{2.5}}$, to address the
endogeneity problem. The estimation results from TSLS are reported in
columns (4)--(7) of Table~\ref{tab:main_results}.

The upper panel of Table~\ref{tab:main_results} in columns (3) and (4) shows
the coefficients of thermal inversion in the first-stage estimation of $%
log(PM_{2.5})$, which suggests that $log(PM_{2.5})$ increases as the
frequency of thermal inversion increases. [\textbf{NOTE}: I think we need
the equation of the first stage which is missing.] This positive
relationship is intuitive based on our discussion on the validity of thermal
inversion as an instrumental variable for $log(PM_{2.5})$ in section~\ref%
{sec:empirical_strategy}: thermal inversion traps air pollution, lifting the
concentration of $\mathrm{PM_{2.5}}$. After being instrumented, the
variation in the fitted value of $log(PM_{2.5})$ in the second stage
regression of exports becomes independent of production activities, and so
the estimation bias is corrected. The Kleibergen-Paap (KP) Wald rk
F-statistic \citep{kleibergen2006generalized} for weak identification test
is much larger than the 10\% critical value (16.38) of Stock-Yogo weak ID
test \citep{stock2005asymptotic}, suggesting that thermal inversions is
strong instrument variable. The lower panel of Columns (4)--(7) report
results of TSLS estimates of regression Model (\ref{equ1}). Column (4)
includes the fitted value of $log(PM_{2.5})$ only, while columns (5)--(7)
include all firm characteristics and weather controls. Columns (5)--(7) are
different in terms of using different measures of productivity: LP-method
TFP in column (5), OP-method TFP in column (6), and labor productivity in
column (7). Comparing the estimates of the $log(PM_{2.5})$ coefficients from
OLS and those from TSLS, we observe that the pollutin effects are
overestimated by OLS regression. Intuitively, this is because air pollution
is positively correlated with exports: more exports and air pollution both
could be a consequence of more production; therefore, when production is
high due to certain factors that are not controlled directly (omitted
variables that included in the error term), these factors drive up air
pollution, which partially offsets the negative effect of air pollution on
exports, rendering the coefficient overestimated (or attenuated).

As the TSLS can address the concerns of endogeneity problem with the OLS
regression, we explain our findings using results from TSL hereafter.

In all TSLS regressions, the estimated coefficients of $log(PM_{2.5})$ are
negative and statistically significant, thereby suggesting that air
pollution reduces the values of firms' exports. In particular, based on
column (5), our result suggests that an one percent increase in $\mathrm{%
PM_{2.5}}$ leads to a 1.09 percent drop in exports. This magnitude is
remarkable, considering that the average annual growth rate of exports in
China was 18.1\% during the sample period. [\textbf{NOTE}: this figure alone
cannot show the economic significance, you should provide the number on the
percentage increase in $\mathrm{PM_{2.5}}$ in China over the sample period.]

We also observe that the magnitude of the coefficient of $log(PM_{2.5})$ is
slightly larger after controlling for firm characteristics, comparing
columns (5)--(7) to column (4). This comparison result implies that air
pollution may affect exports partially through the channel of affecting
firms' productivity and investment decisions and partially through other
channels. The recent literature \citep{fu2021air,khanna2021productivity} has
found that pollution reduces firm productivity. We investigate the working
mechanism of the impact of air pollution on exports in section~\ref%
{sec:mechanism}.

\begin{center}
$<$Table~\ref{tab:main_results} Here$>$
\end{center}

[\textbf{NOTE}: revise this part later] We also check the robustness of our
results to the sample selection by using the sample of exporting firms with
positive export values. In this case, some firms' exports could be
discontinued temporarily for some years. The estimation results are
presented in Appendix~\ref{sec:A.1}. Basically, the magnitude of the $%
log(PM_{2.5})$ coefficients becomes larger, compared with those in Table~\ref%
{tab:main_results}, when we allow temporary discontinuity of exports.
Intuitively, temporary discontinuity \emph{per se} is a strategy of firms in
response to elevated air pollution; therefore, the impact of air pollution
could be larger in this case compared with the case using the sample of
continuous exports for analysis.

\subsection{Effects of Air Pollution on Extensive Margins of Exports}

\label{sec:4.2} In addition to the intensive margins that air pollution
affects exports, it may also influence firms' decisions to enter or exit the
export markets, that is, the entensive margins of exports. We investigate
this issue in this section.

A firm is defined as an entrant to the export market in a year if the firm
has a positive export value in that year but does not have any export in the
previous year. Similarly, a firm is defined as an exiter from the export
market in a year if the firm does not have any export in that year but has a
positive export value in the previous year. Formally, we use binary
variables $Entry$ and $Exit$ to indicate a firm's respective entry and exit
decisions as follows. First, to analyze entry in year $t$, our sample
consists of all firms without exporting in year $t-1$, and we define 
%   Definition of $Entry$ and $Exit$ is in following Eq.(\ref{equ4}) and Eq.(\ref%
%   {equ5}), respectively. An $Entry$ is indicated as 1 if a firm reports no
%   exports in year $t-1$ but started exporting in year $t$. Exit exporters are
%   defined as those that export in year $t-1$ but no exports in year $t$. 
\begin{equation*}
Entry_{ict}=\left\{ \begin{aligned} & 1 & if \ Export_{ict-1} = 0 \ and \
Export_{ict} > 0 \\ & 0 & if \ Export_{ict-1} = 0 \ and \ Export_{ict} = 0.
\end{aligned}\right.
\end{equation*}%
Second, to analyze exit in year $t$, our sample consists of all firms with
exporting in year $t-1$, and we define

\begin{equation*}
Exit_{ict}=\left\{ \begin{aligned} & 1 & if \ Export_{cit-1} > 0 \ and \
Export_{ict} = 0 \\ & 0 & if \ Export_{ict-1} > 0 \ and \ Export_{ict} > 0.
\end{aligned}\right.
\end{equation*}

[\textbf{NOTE}: I correct the definitions above. Please confirm. Do not
italic the words unless they are variables. Please change.]

As the definition of entry and exit in a given period requires the
information of the firm's exporting status in the previous period, the
first-year observation of each firm is dropped from the sample.

We analyze firms' entry and exit decisions using IV-Probit estimation, in
which thermal inversion is still applied as the instrumental variable for
air pollution. [\textbf{NOTE}:\ I think we need to present the model. Or we
need to refer to Model (1). Let me know what we should do.] Basically, we
first run OLS regression of the endogenous variable ($log(PM_{2.5})$) on
instrument variable (thermal inverstion) and exogenous variables and then
save the residuals from this first stage. In the second step, we run the
Probit model of entry and exit, respectively, on the air pollution [\textbf{%
Question}: what is this?] residuals from the first stage, together with
other exogenous variables, to get consistent estimator of air pollution %
\citep{wooldridge2010econometric}. 
%   Note that in $Entry$ case, exporters that were already
%   exporting in year $t-1$ are not included in the sample; in $Exit$ case,
%   firms that were not exporting in year $t-1$ are not included in the sample. 

[\textbf{NOTE}: the following discussions do not seem to be consistent with
the table, regarding different columns, etc. Please \textbf{rewrite} them.]
Table~\ref{tab:entry_exit} reports the marginal effects of the variables in
the Probit model for both entry and exit decisions, which are estimated
using maximum likelihood estimation. The standard errors are estimated by
bootstrap. Columns (1) - (4) report the marginal effects of variables on the
probability of entering or exiting export market. Columns (2) and (4)
include the firm characteristics other than the key variable of our
interest, $log(PM_{2.5})$. Similar to the results from the intensive margin
analysis, the air pollution generates negative effects on the extensive
margins: An increase of $\mathrm{PM_{2.5}}$ concentration reduces the
probability for the firms to enter the export sector, while it increases the
probability for the firms to exit the export sector. All these estimates are
statistically significant at 1\% level. We will explain the reasons in
Section~\ref{sec:mechanism}.

\begin{center}
$<$Table~\ref{tab:entry_exit} Here$>$
\end{center}

\subsection{Robustness Check}

In this section, we check the robustness of our estimation results obtained
above, using county-level aggregate exports and a different instrumental
variable, respectively. [\textbf{NOTE}: we only have two robustness. It
seems too few. Others?]

\subsubsection{County-level Aggregate Exports}

Since the key regressor of our interest, $log(PM_{2.5})$, the weather
controls, and the instrumental variable are all featured with county-level
variations, the impact of air pollution on individual firms' exports could
be attributed to idiosyncratic shocks that share common trends to all firms
in the same county. Because of this concern, we check the robustness of our
results using the county-level export data. First, for intensive margins, we
replace the individual firms' exports in Model (\ref{equ1}) with the
logarithmic average export value of firms in county $c$ in year $t$, denoted
by $log(AExport_{ct})$. Specifically, the everage is calculated based on all
firms with positive export values in the year [\textbf{NOTE}: please confirm
this]. [\textbf{Question}: How about the firm level charateristics? One way
is to drop them, another way is to calculate the average.] For extensive
margins, we replace the individual firms' $Entry_{ict}$ in Model (?) [%
\textbf{NOTE}:\ we shoudl introduce the model before] with the proportion of
new exporters in county $c$ in year $t$, which is defined as the ratio of
the number of new exporters over the number of all exporters, including the
newly added ones [\textbf{NOTE}: please confirm]. Similarly, we replace the
individual firms' $Exit_{ict}$ in Model (?) [\textbf{NOTE}:\ we shoudl
introduce the model before] with the proportion of existing exporters in
county $c$ in year $t$, which is defined as the ratio of the number of
exiting exporters over the number of all exporters, including the existing
ones [\textbf{NOTE}: please confirm]. [\textbf{NOTE}: I found the answer in
a footnote later. The writing is unclear because you give the definition
now, but the give a more precise one later in a footnote. Anyway, I know
what you mean now. But my question is still valid. I have even one more
question: should the denominator be total exporters or total firms. I prefer
the total exporters, because the total firms is too many.] [\textbf{NOTE}:
similar question about firm charateristics.]

Table~\ref{tab:by_county} presents the TSLS estimations results using the
county-level aggregate data. Columns (1) and (2) report the results from the
intensive margin analysis with different model specifications. In both
columns, the coefficient of air pollution is negative and statistically
significant, which is consistent with firm-level results, confirming that $%
\mathrm{PM_{2.5}}$ has negative effects on exports. Moreover, the magnitude
of the coefficient ($-2.8606$ and $-2.5878$, respectively) is much larger
than that obtained from the firm-level analysis (e.g., $-1.0878$ from column
(5) of Table \ref{tab:main_results}), suggesting that a significant portion
of the idiosyncratic shocks to exports may share some comments trends with
air pollution. [\textbf{Question}: is it also due to omitting firm
charateristics? seems not] [NOTE: First, I am not so sure about this
explanation. Second, it is not a good idea to mention regulation now, which
may cause more question here. Considering removing the following part: \emph{%
For example, in the areas with severe pollution, pollution regulation could
be more stringent, which may generate negative effects on export costs %
\citep{cherniwchan2022environmental}. Although this negative effects could
be heterogeneous across firms, depending on their abatement technology, all
firms in the administrative region could share the same trend. Since
firm-level characteristics are important to explain the difference in cost
changes in response to such regulation, most variation in exports could be
explained by firm specific shocks when the firm-level data is employed, but
this variation is attributed to the air pollution effects when county-level
aggregate data is applied, resulting in a larger coefficient of }$PM_{2.5}$%
\emph{.}]

Columns (3) and (4) report the results of the extensive margins with all
controls included. Columns (3) and (4) report the estimation results from
regressions of the number shares of entry and exit firms in the export
sector of a county\footnote{%
Mathematically, the number shares ($S$) of entry ($I$) or exit ($O$) firms
in the export sector of county $c$ of year $t$ are defined as $%
S_{ct}^{x}=N_{ct}^{x}/N_{ct},x=\text{I or O}$, where $N_{ct}^{x}$ is the
number of firms that enter ($x=I$) or exit ($x=O$) county $c$; and $N_{ct}$
is the number of firms in county $c$ of year $t$.}.[\textbf{NOTE}: the word
definition and the math definition in the footnote are not consistent] The
results are also consistent with those obtained based on firm-level entry
and exit.

\begin{center}
$<$Table~\ref{tab:by_county} Here$>$
\end{center}

\subsubsection{Alternative IV: wind direction}

\label{sec:5.1.1}

\section{Heterogeneity}

\label{sec:5} %  \subsection{Heterogenous effects} \label{sec:5.2}
%  For heterogenous effects, we should include (1) High vs. low productivity firms (interaction terms with PM2.5 variable), (2) High and low labor intensity, (3) TCZ vs.non-TCZ

In this section, we explore the heterogenous effects of air pollution on
firms' exports in order to understand more about the effects and obtain some
ideas about the possible channels through which pollution reduces exports.

\subsection{Productivity}

\label{sec:5.2.1} It is well known from the literature that productivity is
a key determinant of firms' exports. Our result from the intensive margin
analysis shows the average effect air pollution on exports. We conjecture
that the effects are different across firms because firms with different
productivity levels have different capacity to respond to pollution. To
investigate this heterogeneity issue, we categorize all firms into two
groups: high- and low-productive firms. We use the median of the
distribution of firm productivity as the threshold to define a dummy
variable of high-productivity ($=1$ if a firm's productivity is higher than
the median; and 0, otherwise). Then, we add a interaction term between $%
Log(PM_{2.5})$ and this high-productivity dummy to our main regression Model
(\ref{equ1}) and then apply TSLS to the firm-level data.

%  \begin{equation}  \label{eq6}
%   \begin{aligned} Log(Export)_{ict} = \beta_{0} + \beta_{1}Log(PM_{2.5})_{ict} + \beta_{3}Productivity\times
%   Log(PM_{2.5})_{ict} \\ + \beta_{4}\textbf{W}_{ict} + \textbf{X}^{'}_{ict}\Phi +
%   \delta_{i} + \lambda_{t} + \epsilon_{ict} \end{aligned}
%   \end{equation}

%   where \textit{Productivity} is indexed either by its revenue-based TFP or by its apparent labor productivity, all in the logarithm form. 

%   Regarding our predictions on air pollution and heterogeneity, the impact of
% $\rm PM_{2.5}$ concentrations ($\beta _{1}$) is expected to be negative, and $\beta
% _{3}$, the coefficient on the interaction term, should be positive: the
% export value elasticity to $\rm PM_{2.5}$ concentrations changes should decrease with
% the firm's performance. 

We report the estimation results in Table~\ref{tab:hetero_tfp}, with the
firm productivity measured using different estimation methods, i.e., the
LP-TFP, OP-TFP, and labor productivity. The results with full controls are
reported in columns (2), (4), and (6), respectively. The estimation results
are consistent, irrespectively to different measures of productivity. The
coefficients of all independent variables are similar to those reported in
Table~\ref{tab:main_results} in terms of the sign and statistical
significance. The positive and significant coefficients of the interaction
term suggest that air pollution generate weaker effects on the exports of
high-productivity than the low-productivity firms, thereby confirming out
conjecture. 
%In the bottom of these three columns, we report how the total effects of air pollution on exports change with productivity: As the productivity increases from its sample mean to the level of mean plus one standard deviation, the marginal effect of air pollution on exports decreases from -.93 to -.81, when LP TFP is used as the productivity measure. 

\begin{center}
$<$Table~\ref{tab:hetero_tfp} Here$>$
\end{center}

\subsection{Labor Intensity}

\label{sec:5.2.2} The literature on environment and productivity %
\citep[e.g.,][]{chang2016particulate,Adhvaryu2022} has documented the
detrimental effects of air pollution on workers. This observation allows us
to hypothesize that air pollution may affect the operation and efficiency of
the firms with higher labor intensities than those with lower labor
intensities, \textit{ceteris paribus}. Accordingly, air pollution will
affect firms' exports differently. Following the previous literature %
\citep{dewenter2001state}, labor intensity can be measured in two different
ways: 1) labors divided by real capital stock; and 2) labors divided by real
valued added. To test the above hypothesis, we first calculate labor
intensity at industry levels, which is the average value of labor
intensities of all firms in the same industry [\textbf{NOTE}: please confirm
whether this is how you define] We use a four-digit CSIC code to define an
industry. We use industry-level labor intensity instead of firm-level labor
intensity to alleviate the concerns with the endogeneity of labor intensity
at the firm level, which may arise when firms strategically change their
usage of labor in response to air quality shocks. [\textbf{Question}: do you
obtain industry level labor intensity from firm level data?] Industry-level
labor intensity is relatively more exogeneous as it reflects the
charateristics of the industry. [\textbf{Question}: is this at
county-industry level, or for the whole country? better give the notation of
this veriable so that it becomes clear.]

To investigate the heterogeneous effects of air pollution on exports over
industries with different labor intensity, we add an interaction term of
labor intensity and $Log(PM_{2.5})$ to regression Model~(\ref{equ1}). Table~%
\ref{tab:hetero_LI} reports the estimation results. Columns (1) and (2)
report the results using the labor/capital ratio as the measure of labor
intensity, while columns (3) and (4) report the results using the
labor/(value added) ratio as the measure of labor intensity. The results
show that the impact of air pollution on exports is larger in more
labor-intensive industries than it is in less labor-intensive industries.
The coefficients of all other independent variables remain similar to those
reported in Table~\ref{tab:main_results}. The coefficient of labor intensity
is positive, suggesting that firms in more labor-intensive industries are
exporting more than those in less labor-intensive industries, which is
consistent with the comparative advantages of China.

%  \begin{equation}
%   \begin{aligned} Log(Export)_{ict} = \beta_{0} + \beta_{1}Log(PM_{2.5})_{ict} +
%   \beta_{2}Labor Intensity_{ict}  + \beta_{3}Labor Intensity\times \\
%   Log(PM_{2.5})_{ict}  + \beta_{4}\textbf{W}_{ict} + \textbf{X}^{'}_{ict}\Phi +
%   \delta_{i} + \lambda_{t} + \epsilon_{ict} \end{aligned}
%   \end{equation}

% where \textit{Labor Intensity} is calculated by different measures, labors divided by real capital stock and labors divided by real valued added \citep{dewenter2001state}. The former indicates the relative use of factors and the latter measures the labor intensity of production. $Labor Intensity_{ict}$ is in the natural logarithm form. Industry labor intensity is calculated at 4-digit level.

\begin{center}
$<$Table~\ref{tab:hetero_LI} Here$>$
\end{center}

\subsection{Pollution Intensity}

\label{sec:5.2.3} Industries differ significantly in their pollutant
emissions. Manufacturers of chemical products, such as chemical fibers and
non-metallic mineral products, generate much more air pollutants than those
producing scientific and mathematical instruments. When the government sets
environmental policies and implements the regulations, firms in high
pollution-intensive industries are expected to make more investment in
abatement technologies than those in low pollution-intensive industries. The
cost of abatment technologies is a fixed cost to a firm in the short term.
However, as air condition deteriorates, the marginal costs due to additional
abatement needs may increase and the increase is larger for the firms in the
pollution-intensive industries than others in the less-pollution-intensive
industries, thereby resulting in possible heterogenous effects of air
pollution on exports. We consider pollution intensity at industry level
because firm level pollution data is not available. Also due to data
availability, we do not have industrial pollution intensity for each county
and thus, it is the country average industrial pollution intensity.

We investigate the heterogeneous effects of air pollution due to different
pollution intensities by estimating Model~(\ref{equ1}) with an additional
interaction term of an industrial pollution-intensity variable and air
pollution $log(PM_{2.5})$. We define industrial pollution-intensity
variable, denoted by $PI_{pkt}$, as the ratio of total emissions of
pollutant $p$ from industrial $k$ (at 2-digit CSIC level) to the total value
of output of the industry, in year $t$. We use four pollutants to measure
pollution emissions, respectively, which are total industrial waste gas,
sulfur dioxide (SO$_{2}$), soot smoke, and suspended particulate matter. The
suspended matter includes coarse particles with a diameter of 10 micrometers
or less, which covers a broader range of particulate matters including $%
\mathrm{PM_{2}.5}$. We obtain the value for $PI_{pkt}$ based on our
calcuation from China Environment Yearbooks (1999-2008) published by the
Ministry of Environmental Protection. [\textbf{NOTE}: We could do a better
job here because this data is not at county level. For example, if a county
does not have any production which generates a particular type of pollutant,
then the variable is not relevant for the county. Can we use industry
composition of a county to infer that?]. We run the TSLS regression using
the four types of $PI_{pkt}$, respectively, with $PI_{pkt}$ in the form of
natural logarithm, i.e., $Log(PI_{pkt})$.

Table~\ref{tab:hetero_PI} reports the estimation results. Columns (1)--(4)
correspond to different pollution intensities measured by by total
industrial waste gas,\text{ }SO$_{2}$, soot smoke, and suspended particulate
matter, respectively. The coefficients of the interaction term of $%
Log(PI_{pkt})\times Log(PM_{2.5})$ are all negative and statistically
significant, thereby suggesting that the negative impacts of air pollution
on exports are stronger for firms in pollution-intensive industries than
others. The estimated coefficient of $Log(PI_{pkt})$ is positive, thereby
implying that pollution-intensive industries in China have more exports. [%
\textbf{NOTE}: there are other results in the table. should mention them.]

\begin{center}
$<$Table~\ref{tab:hetero_PI} Here$>$
\end{center}

\subsection{Environmental Regulation}

\label{sec:5.2.4} Stringer environmental regulations may affect firms'
production and exports through increasing the firms' abatement costs or
inducing the firms to reduce production in order to comply with the
regulations. Therefore, we conjecture that pollution affects firms' exports
more in regions where enviroinmental regulations and enforcement are tougher
compared to other regions. To test this hypothesis, we need information
about envirionmental regulations and enforcement at regional level, as
opposed to country level. Although documents of national and regional
environmental policies and regulations are exist, it is extremely difficult
to conduct text analysis to quantify the stringency of those regulations
across regions. However, there exists a national policy that divides all
regions to two zones, each facing different levels of envrionmental targets,
which is called the Two Control Zones (TCZ) policy. We test the above
hypothesis by making use of the TCZ policy.

In 1998, the State Council of China issued a policy in \textquotedblleft The
Official Reply of the State Council Concerning Acid Rain Control Areas and SO%
$_{2}$ Pollution Control Areas\textquotedblright . Since then, there has
been tightened control of acid rain and the emission of SO$_{2}$ in the
targeted areas, which are named as TCZ because of the control of the two
types of pollution. TCZ includes 175 prefecture-level cities, accounting for
11.4\% of the nation's territory, 40.6\% of the population, 62.4\% of GDP,
and 58.9\% of total SO$_{2}$ emissions in 1995 \citep{hao2001plotting}.
While the policy sets clear environmental targets for TCZ cities, the other
cities, called non-TCZ cities, 205 in total, are not required to meet those
targets. Although the TCZ policy sets targets only for acid rain and SO$_{2}$
whereas our study is about $\mathrm{PM_{2.5}}$, [\textbf{Question}: do we
have SO2 data to check their correction with PM2.5? Any papers talking about
their correlation? ] we believe governments in TCZ enforce gereral pollution
controls more strictly than those in non-TCZ. Hence, firms in TCZ policy as
a quasi-natural experiment for analyzing the heterogeneity of the negative
effects of air pollution on exports caused by differential environmental
regulations. \textbf{HERE}

We add an interaction term of a binary variable $TCZ_{c}$ ($=1$ if the
county is in TCZ; $0$ otherwise) and $PM_{2.5}$ to regression~\ref{equ1}.
The estimation results are reported in Table~\ref{tab:hetero_cost}. Column
(1) reports the estimation results using the full sample, assuming that the
marginal effects of all variables other than air pollution are the same in
both TCZ and non-TCZ. The coefficient of the interaction term of $%
TCZ_{c}\times log(PM_{2.5})$ [\textbf{Question}: should we include subscript 
$ct$ for $log(PM_{2.5})$\ here and also in other places?] is negative and
statistically significant at 10\% significance level, suggesting that air
pollution has a larger adverse effect on exports in TCZ than it has in
non-TCZ. In addition to the interaction term, we have two observations.
First, the estimated coefficient of the independent term $TCZ_{c}$ is
positive and significant, indicating that firms in TCZ export more than
others in non-TCZ. Second, the estimated coefficient of the independent term 
$log(PM_{2.5})$ becomes statistically insignificant, that is, air pollution
has no significant effect on exports in non-TCZ, where $TCZ_{c}=0$. While
the former observation is expected, the latter one is surprising.

Columns (2)--(3) report the results using the sub-samples of observations in
TCZ and non-TCZ, respectively, without imposing equality constraints on
their coefficients. The results are consistent with those in column (1).

\begin{center}
$<$Table~\ref{tab:hetero_cost} Here$>$
\end{center}

\section{Mechanisms}

\end{document}
